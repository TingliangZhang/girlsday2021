% Compile with latex+dvipdfmx, pdflatex, xelatex or lualatex
% XeLaTeX is recommanded

\documentclass[UTF8]{ctexart}
\usepackage[T1]{fontenc}
\usepackage[colorlinks,
linkcolor=red,
anchorcolor=blue,
citecolor=green
]{hyperref}
\usepackage{tcolorbox}

\title{iTag-从见面开始的微微熟社交生态}
\author{张庭梁}
\date{\today}

\begin{document}

\maketitle

\section{现有产品研究}
本章我们研究现有的陌生人社交和部分典型的熟人社交软件。

\subsection{陌生人社交定义}

\begin{tcolorbox}
    互联网语境下的社交是用户间的信息交流和互动,最典型的社交产品是早期的微信和QQ,但现在微信和QQ已不仅仅是社交产品,更是生活方式。

    在社交的分类中,按用户关系距离来分,可分为熟人社交(QQ、微信)和陌生人社交(探探、陌陌、Soul、她说等婚恋交友软件)。

    从社交的定义可以得到陌生人社交指陌生用户间的信息交流和互动,社交整体过程归纳总结起来就是以下两点:关系的建立+关系的沉淀(维系和发展)。而陌生人社交的重点会更偏关系的建立,特点是:它是对现实的补充,具有很大的虚拟性。\cite{StrangerDefine}
\end{tcolorbox}

其实建立关系只是第一步,更重要的是关系的沉淀。我们希望开发的是一款提供相识机会,但更注重引导关系发展的社交应用。

\subsection{Tinder}

\begin{tcolorbox}
    Features
    \begin{enumerate}
        \item Swipe is central to Tinder's design. The app's algorithm provides users, swipe right to "like" potential matches and swipe left to continue on their search from the compatible matches.
        \item Messaging is also a heavily utilized feature. Once a user matches with another user, they're able to exchange text messages on the app.
        \item Instagram integration enables users to access other users' Instagram profiles.
        \item Common Connections allows users to see whether they share a mutual Facebook friend with a match (a first-degree connection on Tinder) or when a user and their match have two separate friends who happen to be friends with each other (considered second-degree on Tinder).
        \item Tinder Gold, introduced worldwide in August 2017, is a premium subscription feature that allows the user to see those who have already liked them before swiping.
        \item Panic button is introduced in the US from January 2020. The feature will include emergency assistance, location tracking, and photo verification.\cite{WikiTinderFeatures}
    \end{enumerate}
\end{tcolorbox}

\section{参考文献}

\bibliography{social}
\bibliographystyle{plain}

\end{document}