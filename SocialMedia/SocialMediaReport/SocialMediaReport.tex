% Compile with latex+dvipdfmx, pdflatex, xelatex or lualatex
% XeLaTeX is recommanded

\documentclass[UTF8]{ctexart}
\usepackage[T1]{fontenc}

\title{iTag-从见面开始的微微熟社交生态}
\author{张庭梁}
\date{\today}

\begin{document}

\maketitle

\section{现有产品研究}
本章我们研究现有的陌生人社交和部分典型的熟人社交软件。

\subsection{陌生人社交定义}
互联网语境下的社交是用户间的信息交流和互动,最典型的社交产品是早期的微信和QQ,但现在微信和QQ已不仅仅是社交产品,更是生活方式。

在社交的分类中,按用户关系距离来分,可分为熟人社交(QQ、微信)和陌生人社交(探探、陌陌、Soul、她说等婚恋交友软件)。

从社交的定义可以得到陌生人社交指陌生用户间的信息交流和互动,社交整体过程归纳总结起来就是以下两点:关系的建立+关系的沉淀(维系和发展)。而陌生人社交的重点会更偏关系的建立,特点是:它是对现实的补充,具有很大的虚拟性。

\end{document}