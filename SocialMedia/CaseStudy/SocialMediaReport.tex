% Compile with latex+dvipdfmx, pdflatex, xelatex or lualatex
% XeLaTeX is recommanded

\documentclass[UTF8]{ctexart}
\usepackage[T1]{fontenc}
\usepackage[colorlinks,
linkcolor=red,
anchorcolor=blue,
citecolor=green
]{hyperref}
\usepackage{tcolorbox}
\tcbuselibrary{breakable}
% 如果框内的内容多于一页需要设置自动断页[breakable]但是要注意先使用breakable库

\title{陌生人社交生态概述}
\author{张庭梁}
\date{\today}

\begin{document}

\maketitle

\section{摘要}

本文从探探的发展出发,延伸探讨广义上的陌生人社交软件的发展和存在的问题,最后探讨现有软件的问题成因和可能的解决方案。

\section{陌生人社交软件发展-以探探为例}
本章以探探为例,简单梳理陌生人社交软件发展。

探探是中国大陆一款基于地理位置的社交应用程序,由王宇和潘滢创立,于2014年上线,其目标人群定位在20岁至26岁的用户。使用上,探探被认为模仿了同类应用程序Tinder,双方只有互相喜欢,才能开始聊天,因此得名“约炮神器”。探探亦上线了“擦肩而过”和“匿名表白”功能,并做了一系列尝试来规范用户行为。探探推出后在85后到90后的用户群中较为热门,但其“匿名表白”功能曾引发侵权争议,其安全性也遭到过质疑。

2018年2月23日陌陌斥资7.71亿美元收购探探科技(北京)有限公司。交易成功后,探探团队继续独立运营产品和品牌。\cite{MoMoHistory}

探探被认为模仿了美国同类应用程序Tinder,其主要基于地理位置和大数据平台,向用户提供个性化推荐。其在交互方式上采用了简单的动作设置:如果喜欢某一对象,就往右滑,不喜欢则往左滑。之后,系统自动进行配对,如果双方互相喜欢,系统会明确告知给两个用户,双方才能开始聊天,否则彼此不会收到任何消息。

此外,探探亦上线了“擦肩而过”和“匿名表白”功能,“擦肩而过”是探探开发的一项基于GPS系统的服务,如果有喜欢的人从附近经过,探探会进行提示并显示距离。“匿名表白”是探探基于手机通信录的一种匿名表达爱慕之情的功能,使用该功能,用户可以对通信录的好友匿名表白,并邀请该位好友加入成为探探用户。在使用“匿名表白”功能时,如果被表白的一方尚未注册探探账户,系统会自动发送暗恋短信到被表白一方的手机上,此举引发了侵权争议,存在违法嫌疑。

信息安全性方面,企业家拉里·萨利布拉使用Xcode对软件进行逆向工程时发现该软件缺乏加密措施,因此,稍有技术的黑客即可获取用户的用户名、密码、电话号码,甚至对话。

2019年06月28日,国家网信办发布,其会同有关部门,针对网络音频乱象启动专项整治行动。根据群众举报线索,经核查取证,首批依法依规对吱呀、Soul、语玩、一说FM等26款传播历史虚无主义、淫秽色情内容的违法违规音频平台,分别采取了约谈、下架、关停服务等阶梯处罚,对音频行业进行全面集中整治。有关负责人表示,此次专项整治不仅要坚决有效遏制行业乱象,也要积极规范行业发展,促进网络生态持续向好。\cite{Audio}

根据《关于开展App违法违规收集使用个人信息专项治理的公告》,受中央网信办、工信部、公安部、市场监管总局委托,全国信息安全标准化技术委员会、中国消费者协会、中国互联网协会、中国网络空间安全协会成立App专项治理工作组,对用户数量大、与民众生活密切相关的App隐私政策和个人信息收集使用情况进行评估。2019年7月12日,包括探探等在内的30款App因违反《网络安全法》关于收集使用个人信息的相关规定,被通报30日内完成整改。\cite{Privacy}

探探部分用户协议摘录如下:

\begin{tcolorbox}
    3.1 为了能使用本服务,您需要注册探探帐号,应当使用手机号码绑定注册,请用户使用尚未与“探探”帐号绑定的手机号码,以及未被探探文化封禁的手机号码注册“探探”帐号。根据法律法规的要求,探探文化可以根据用户需求或产品需要对帐号注册和绑定的方式进行变更,您应在法律法规的要求下予以配合。

    3.2 “探探”系基于地理位置的移动社交产品,用户注册时应当授权探探文化公开及使用其地理位置信息方可成功注册“探探”帐号。故用户完成注册即表明用户同意探探文化提取、公开及使用用户的地理位置信息。

    3.3 鉴于探探帐号的注册方式,且为了保护您账号的安全,您同意探探文化在注册时使用您提供的手机号码及/或自动提取您的手机号码及您的手机设备识别码等信息用于注册。同时如果您可以选择开启使用某些功能,您同意探探文化取得您的手机通讯录信息以使这些功能可以实现。

    3.4 在用户注册及使用本服务时,探探文化需要搜集能识别用户身份的个人信息以便探探文化可以在必要时联系用户,或为用户提供更好的使用体验。探探文化搜集的信息包括但不限于用户的姓名、性别、年龄、出生日期、身份证号、地址、学校情况、公司情况、所属行业、兴趣爱好、常出没的地方、个人说明;探探文化同意对这些信息的使用将受限于个人隐私信息相关法律和探探隐私政策的约束。\cite{MoMoAgreement}
\end{tcolorbox}

其实和国内很多软件一样,在安装时索取了远超自己APP需要的权限,当然这和隐私保护的乱象有关,同样的应用在苹果商店就老实很多,比安卓商店索取的权限要少很多。

\section{现有产品研究}
本章我们研究现有的陌生人社交和部分典型的熟人社交软件。

\subsection{陌生人社交定义}

\begin{tcolorbox}
    互联网语境下的社交是用户间的信息交流和互动,最典型的社交产品是早期的微信和QQ,但现在微信和QQ已不仅仅是社交产品,更是生活方式。

    在社交的分类中,按用户关系距离来分,可分为熟人社交(QQ、微信)和陌生人社交(探探、陌陌、Soul、她说等婚恋交友软件)。

    从社交的定义可以得到陌生人社交指陌生用户间的信息交流和互动,社交整体过程归纳总结起来就是以下两点:关系的建立+关系的沉淀(维系和发展)。而陌生人社交的重点会更偏关系的建立,特点是:它是对现实的补充,具有很大的虚拟性。\cite{StrangerDefine}
\end{tcolorbox}

其实建立关系只是第一步,更重要的是关系的沉淀。我们希望开发的是一款提供相识机会,但更注重引导关系发展的社交应用。

\subsection{陌生人社交平台分类}
我将陌生人社交平台分为三类:LBS(Location Based Services,基于位置的服务)社交应用、婚恋市场、兴趣交友。

LBS社交包括Tinder、探探和陌陌、微信附近的人等,主打短期关系和荷尔蒙社交,颜值基本上就是其中唯一的社交资本。

婚恋市场主要玩家为世纪佳缘、百合网,社交资本稍多元,包括收入、职业、户口等,但婚恋市场更加功利,用户普遍为30-40岁,有更高的收入,追求相亲的效率,VIP服务收费较高,从几万到几十万不等。

兴趣交友类包括Soul、Summer等,本意是兴趣交友,社交资本更为多元,有助于寻找到共同话题。Summer主打高校市场:分享校园欢乐日常、大学生专属身份认证、面向全国高校开放、答题交友。Soul则主打匿名性的轻松感,不用上传头像、视频露脸或者实名。但是这些应用经过一年多的发展,鱼龙混杂的用户涌入,渐渐和探探等同质化。

Tinder的核心是左滑跳过右滑Like的配对机制,只有双方互相Like之后才能开始进一步聊天,后续Tinder还推出了和Facebook Instagram等社交软件的联动,付费版本可以Like更多的人甚至在Like之前看这个人有没有Like你,在2020年因为安全方面的考量,Tinder也推出了一键SOS报警功能。\cite{WikiTinderFeatures}
在信息安全方面,Tinder也曾爆出不少漏洞,每次打开和进行匹配的精确地理位置甚至聊天视频内容的泄露漏洞,以及卫报记者单人长达800页的用户数据历史,包括时间、地点、个人喜好、受众人群、每个照片浏览时间。一旦这些十分隐私的详尽数据泄露,后果不堪设想。\cite{WikiTinderFeatures}

探探应用中存在大量虚假信息、骗子、甚至聊天机器人。打擦边球构筑的不健康的商业模式。用户粘性有限,商业潜力有限。软件缺乏加密措施,因此,稍有技术的黑客即可获取用户的用户名、密码、电话号码,甚至对话。\cite{WikiTantan}
陌陌类应用其实依靠的是打擦边球构筑的不健康的商业模式,用户粘性和商业潜力有限。其中大量虚假信息、骗子、甚至聊天机器人,进一步增加了用户对其不信任感。\cite{MoMoWhy}

以世纪佳缘、百合网等交友软件为代表的婚恋市场婚恋市场则更加功利,用户普遍为30-40岁,有更高的收入,追求相亲的效率。
但是婚恋网站上面信息不实,婚介和外包由于资源不足合伙骗人,甚至还有金融理财诈骗、传销组织奔现等陷阱,这些因素让婚恋市场的信誉不断受损。\cite{CheatTheLiving}
知乎上也有大量的关于平台或代理商请的托恶意骗钱行为的例子,比如\url{https://www.zhihu.com/question/37751869}
婚恋平台上存在大量不实个人信息但是过于严格的筛查审核机制会劝退很多用户,用户不愿意一次性上传过多真实个人信息和照片,如学位证身份证等。\cite{WikiJiayuan}
盈利主要靠VIP或一对一红娘收费,收费普遍极高,条件差距大的收费更高,甚至于几十万的天价中介费。盈利方式不健康。

SOUL这一兴趣交友网站开始的模式很吸引人,类似树洞的匿名,个人信息的自由,和探探等yp应用区分开了。但是没有树洞的自由,屏蔽词恶心用户。匿名下的乱象丛生令这个平台鱼龙混杂,广场变为寻求满足感的匿名微博式平台。\cite{Soul}
Summer类似Soul但是主打高校市场,核心功能包括分享校园欢乐日常、大学生专属身份认证、面向全国高校开放、答题交友趣味刷题。但近些年发展也不算很快。

\section{总结与讨论}

\section{参考文献}

\bibliography{social}
\bibliographystyle{plain}

\end{document}